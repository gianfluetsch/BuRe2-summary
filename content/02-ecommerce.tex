\section{E-Commerce}

\subsection{Zustandekommen eines Vertrages im Internet}
\begin{itemize}
    \item \textbf{handlungsfähige Parteien (ZGB)} $\rightarrow$ volljährig ( > 18 Jahre) + urteilsfähig
    \item \textbf{Konsens} $\rightarrow$ übereinstimmende gegenseitige Willensäusserung
    \item \textbf{alle objektiv und subjektiv wesentlichen Vertragspunkte (OR)}
    \begin{itemize}
        \item Kaufvertrag (Käufer / Verkäufer, Produkt, Preis)
        \item \textit{objektiv} = z.B. welches Produkt zu welchem Preis
        \item \textit{subjektiv} = z.B. was für eine Farbe hat ein Produkt (wichtig für Käufer, weniger wichtig für Verkäufer)
    \end{itemize}
    \item Antrag / Annahme = \textbf{Vertragsschluss} (OR 3)
\end{itemize}

\subsubsection{Vertragsschluss im Internet}
\begin{itemize}
    \item handlungsfähige Parteien / Konsens über alle wesentlichen Punkte
    \item grundsätzlich formlos gültig (OR)
    \begin{itemize}
        \item falls Schriftlichkeit (OR 12) erforderlich $\rightarrow$ digitale Signatur
        \item Bestellungen sind formlos gültig (digitale Signatur)
    \end{itemize}
    \item i.d.R. unverbindliche Einladung zur Antragsstellung (OR)
    \begin{itemize}
        \item Ausnahme: Internet-Auktion, Download von Programmen etc.
    \end{itemize}
    \item i.d.R. Vertrag unter Abwesenden (OR)
    \begin{itemize}
        \item keine unmittelbare Reaktion (von Person zu Person) auf Erklärung
        \begin{itemize}
            \item Vertrag unter "Abwesenden" (keine direkte Reaktion von Käufer \& Online Shop $\rightarrow$ Abwesenheitsverhältnis)
        \end{itemize}
        \item Entscheidung des Adressaten der Willenserklärung über Annahme des Angebotes „innert angemessener Frist“
        \begin{itemize}
            \item Online Shop muss z.B. innerhalb von 24h die Bestellung/ den Kauf bestätigen
        \end{itemize}
        \item Ausnahme: Anwesenheitsverhältnis (z.B. Chat-Room-Situation oder Internettelephonie)
        \begin{itemize}
            \item Eingangsbestätigung Bestellung und Versandbestätigung muss klar Formuliert sein, damit diese nicht vom Kunden falsch verstanden wird
            \item Online Shop kann bei Verfügbarkeitsanzeigen auf diese Anzahl behaftet werden
        \end{itemize}
    \end{itemize}
\end{itemize}

\subsection{AGB}

\subsubsection{Verbindlichkeit von AGB}
\begin{itemize}
    \item Vertragsbestandteil
    \begin{itemize}
        \item Möglichkeit der Kenntnisnahme
        \item Verständlichkeit und Lesbarkeit
        \item ausdrückliche oder stillschweigende Übernahme (Konsens)
    \end{itemize}
    \item branchenspezifische AGB
    \begin{itemize}
        \item z.B. allg. Bedingungen für Bauarbeiten (SIA-Normen)
    \end{itemize}
\end{itemize}

\subsubsection{Inhalt von AGB}
\begin{minipage}{0.5\linewidth}
    \begin{itemize}
        \item Zeitpunkt des Vertragsschlusses
        \item Widerrufsmöglichkeit
        \item Zahlungsbedingungen
        \item Lieferbedingungen
    \end{itemize}
\end{minipage}
\begin{minipage}{0.5\linewidth}
    \begin{itemize}
        \item Gewährleistung / Garantie / Haftung
        \item Gerichtsstand
        \item anwendbares Recht
    \end{itemize}
\end{minipage}

\subsubsection{Gefahr von AGB}
\begin{itemize}
    \item Zustimmung zu AGB häufig ohne Kenntnisnahme von deren Inhalt (sog. Globalübernahme)
    \item einseitige Verteilung von Rechten und Pflichten zuungunsten der anderen Partei (z.B. Wegbedingung der Haftung, Verrechnungsverzicht etc.)
    \item unklare und ungewöhnliche Regelungen („Überrumpelung“)
    \item „Take-it-or-leave-it-contracts“, wenn ganze Wirtschaftszweige ihre AGB untereinander absprechen
\end{itemize}

\subsubsection{Grenzen für AGB}
\begin{itemize}
    \item abweichende Individualabreden haben Vorrang
    \item zwingendes Recht, gute Sitten, Persönlichkeitsrecht
    \item Verwendung missbräuchlicher Geschäftsbedingungen
\end{itemize}

\subsubsection{Auslegung von AGB}
\begin{itemize}
    \item grundsätzlich wie individuell verfasste Abreden (Willens- und Vertrauensprinzip)
    \item unklare AGB-Bestimmungen werden im Zweifelsfall zuungunsten des Verfassers ausgelegt (sog. Unklarheitenregel)
    \item AGB, mit deren Inhalt Zustimmender nicht rechnen muss, gelten nicht (sog. Ungewöhnlichkeitsregel)
\end{itemize}

\subsubsection{Anforderungen an AGB}
\begin{itemize}
    \item Wie leicht lassen sich die AGB auffinden und ausdrucken?
    \item Ist der Zeitpunkt des \textit{Vertragsschlusses} geregelt?
    \item Besteht die Möglichkeit, einen Bestellung zu widerrufen?
    \item Äussern sich die AGB zu \textit{Zahlungs- und Lieferbedingungen}?
    \item Was gilt bezüglich Gewährleistung / Garantie / Haftung?
    \item Sind ein Gerichtsstand und das anwendbare Recht für den Fall von Streitigkeiten geregelt?
    \item Sind die AGB frei von \textit{widerrechtlichen, irreführenden, ungewöhnlichen und unklaren Bestimmungen}?
\end{itemize}

\subsection{Verträge im internationalen Verhältnis}

\subsubsection{Internationales Privatrecht}
\begin{itemize}
    \item Voraussetzung: internationaler Sachverhalt
    \begin{itemize}
        \item Sachverhalt mit Bezug zum Ausland
        \item Bsp.: Verkäufer hat Sitz in CH / Käufer wohnt in D
    \end{itemize}
    \item Kollisionsrecht
    \begin{itemize}
        \item Bundesgesetz über das Internationale Privatrecht (IPRG)
        \item bilaterale oder multilaterale Staatsverträge (haben stets Vorrang gegenüber dem IPRG)
    \end{itemize}
\end{itemize}