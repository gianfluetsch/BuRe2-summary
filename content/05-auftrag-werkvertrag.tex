\section{Auftrag \& Werkvertrag}

\subsection{Werkvertrag}
Allgemeine Infos zu Werkvertrag $\rightarrow$ siehe \ref{arbeitsrecht:werkvertrag}.

\subsubsection{Gegenstand}
Durch den Werkvertrag verpflichtet sich der \underline{Unternehmer} zur \underline{Herstellung eines Werkes} und der \underline{Besteller} zur Leistung einer Vergütung.

\begin{itemize}
    \item Der Werkvertrag kann formfrei geschlossen werden.
    \item Gegenstand eines Werkvertrages kann die Erstellung aber auch die Veränderung (z.B. Reparatur, Bearbeitung, Veredelung) einer Sache sein.
\end{itemize}

\subsubsection{Vergütung}
\begin{itemize}
    \item Fixpreis
    \begin{itemize}
        \item Wurde eine bestimmte Vergütung vereinbart, gilt diese grundsätzlich unabhängig vom Aufwand des Unternehmers (+ oder - ). Nur bei ausserordentlichen, unvorhersehbaren Umständen, welche die Fertigstellung des Werkes hindern oder übermässig Erschweren, kann der Richter nach seinem Ermessen eine Erhöhung des Preises oder die Auflösung des Vertrages bewilligen
    \end{itemize}
    \item Keine oder nur ungefähre Preisvereinbarung
    \begin{itemize}
        \item Fehlt im Vertrag eine Vereinbarung über die Vergütung oder ist sie sehr unbestimmt, so wird der Werklohn nach Massgabe des Wertes der Arbeit und der Aufwendungen des Unternehmers festgesetzt.
        \item Überschreitung des Kostenansatzes nach Art. 375 OR: Wird ein mit dem Unternehmer verabredeter ungefährer Ansatz ohne Zutun des Bestellers unverhältnismässig überschritten, so hat dieser sowohl während als auch nach der Ausführung des Werkes das Recht, vom Vertrag zurückzutreten.
        \item \textcolor{red}{Als Faustregel gilt eine Überschreitung von 10\%}
    \end{itemize}
    \item Vergütung nach der SIA - Norm 118 (ähnlich wie AGB)
    \begin{itemize}
        \item Enthält Regeln betreffend \underline{Abschluss, Inhalt} und \underline{Abwicklung} von Verträgen über Bauarbeiten
        \item Ist nur anwendbar, wenn die Vertragsparteien ihre Übernahme ausdrücklich oder stillschweigend vereinbart haben
    \end{itemize}
\end{itemize}

\subsubsection{Folgen bei Vertragsverletzung}
\begin{itemize}
    \item Der Unternehmer muss \underline{verschuldensunabhängig} für Werkmängel einstehen.
    \item Die Ausübung der Mängelrechte ist aber an folgende Voraussetzungen geknüpft:
    \begin{itemize}
        \item Werk ist bei Ablieferung mangelhaft (Abweichung Ist- von der Soll-Beschaffenheit)
        \item Mangel muss schon zu Beginn vorhanden gewesen sein (muss aber nicht direkt sichtbar sein)\\ $\rightarrow$ Schimmel weil nie gelüftet wird ist \underline{KEIN} Mangel
        \item Besteller erfüllt Prüf- und Rügeobliegenheit
    \end{itemize}
    \item Der Anspruch ist nicht verjährt (2 - 5 Jahre).
    \item Verzug
    \begin{itemize}
        \item  Befindet sich der Unternehmer im Verzug, so kann der Besteller vorzeitig vom Vertrag zurücktreten (Gestaltungsrecht).
    \end{itemize}
\end{itemize}

\paragraph{Mängelhaftung}
\begin{itemize}
    \item \textbf{Wandelung}
    \begin{itemize}
        \item Die Wandelung kann erklärt werden, wenn das hergestellte Werk an \underline{erheblichen Mängeln} leidet oder so sehr vom Vertrag abweicht, dass es für den Besteller \underline{unbrauchbar} ist oder ihm nicht zugemutet werden kann, das Werk anzunehmen.
    \end{itemize}
    \item \textbf{Minderung}
    \begin{itemize}
        \item Der Umfang der Minderung ist nach der \underline{relativen Methode} zu berechnen. Danach ist die volle Vergütung in dem Verhältnis zu kürzen, in dem der Wert des Werkes im mangelfreien Zustand zum Wert des mangelhaften Werkes steht.
    \end{itemize}
    \item \textbf{Nachbesserung}
    \begin{itemize}
        \item Die Nachbesserung kann verlangt werden, sofern diese möglich ist und dem Werkunternehmer nicht übermässige Kosten verursacht. Die Nachbesserungskosten sind dann übermässig, wenn sie zum Nutzen, den die Mängelbeseitigung dem Besteller bringt, in einem Missverhältnis stehen, d.h. die Nachbesserungskosten unverhältnismässig hoch wären.
    \end{itemize}
    \item \textbf{Schadenersatz}
    \begin{itemize}
        \item Kumulativ zur Wandlung, Minderung oder Nachbesserung kann beim Vorliegen von vermutetem Verschulden (Beweislastumkehr) Schadenersatz verlangt werden.
    \end{itemize}
\end{itemize}

\subsubsection{Betreffend den Stoff}
\begin{itemize}
    \item Soweit der Unternehmer die Lieferung des Stoffes übernommen hat, haftet er dem Besteller für die Güte desselben und hat Gewähr zu leisten \underline{wie ein Verkäufer} (Sachgewährleistung).
    \item Den vom Besteller gelieferten Stoff hat der Unternehmer \underline{mit aller Sorgfalt} zu behandeln, über dessen Verwendung Rechenschaft abzulegen und einen allfälligen Rest dem Besteller zurückzugeben.
    \item Zeigen sich bei der Ausführung des Werkes Mängel an dem vom Besteller gelieferten Stoffe oder an dem angewiesenen Baugrunde, oder ergeben sich sonst Verhältnisse, die eine gehörige oder rechtzeitige Ausführung des Werkes gefährden, so hat der Unternehmer dem Besteller \underline{ohne Verzug davon Anzeige zu machen}, widrigenfalls die nachteiligen Folgen ihm selbst zur Last fallen (Orientierungspflicht).
\end{itemize}

\subsubsection{Vorzeitiger Rückzug}
Solange das Werk unvollendet ist, kann der Besteller gegen Vergütung der bereits geleisteten Arbeit und gegen volle Schadloshaltung des Unternehmers \underline{jederzeit} vom Vertrag zurücktreten.

\subsection{Auftrag}
Allgemeine Infos zum Auftrag $\rightarrow$ siehe \ref{arbeitsrecht:auftrag}.

\subsubsection{Übersicht}
\begin{itemize}
    \item \textbf{Einfacher Auftrag}
    \item \textbf{Auftrag zur Ehe- oder zur Partnerschaftsvermittlung}
    \begin{itemize}
        \item Wer einen Auftrag zur Ehe- oder zur Partnerschaftsvermittlung annimmt, verpflichtet sich, dem Auftraggeber gegen eine Vergütung Personen für die Ehe oder für eine feste Partnerschaft zu vermitteln
    \end{itemize}
    \item \textbf{Der Kreditbrief und der Kreditauftrag}
    \begin{itemize}
        \item Der Kreditauftrag verpflichtet den Beauftragten, einem Dritten Kredit zu gewähren.\newpage
    \end{itemize}
    \item \textbf{Der Mäklervertrag}
    \begin{itemize}
        \item Durch den Mäklervertrag erhält der Mäkler den Auftrag, gegen eine Vergütung, Gelegenheit zum Abschlusse eines Vertrages nachzuweisen oder den Abschluss eines Vertrages zu vermitteln.
    \end{itemize}
    \item \textbf{Der Agenturvertrag}
    \begin{itemize}
        \item Agent ist, wer die Verpflichtung übernimmt, dauernd für einen oder mehrere Auftraggeber Geschäfte zu vermitteln oder in ihrem Namen und für ihre Rechnung abzuschliessen, ohne zu den Auftraggebern in einem Arbeitsverhältnis zu stehen.
    \end{itemize}
\end{itemize}

\subsubsection{Auftrag - Gegenstand}
Durch die Annahme eines Auftrages verpflichtet sich der \underline{Beauftragte}, die ihm übertragenen Geschäfte oder Dienste vertragsgemäss zu besorgen.

\begin{itemize}
    \item Kann formfrei geschlossen werden
    \item Beauftragter hat den Auftrag vertragsgemäss auszuführen
    \item kein Erfolg geschuldet
    \item Gegenstand eines Auftrages kann jede beliebige \underline{persönliche Handlung} sein. Voraussetzung für das Vorliegen eines Auftrages ist stets, dass es sich um ein \underline{Tätigwerden in fremdem Interesse} handelt.
    \item z.B. Anwalt anheuern ist ein Auftrag (Anwalt schuldet keinen Erfolg)
\end{itemize}

\subsubsection{Auftrag - Sorgfaltspflicht}
\begin{itemize}
    \item Der Beauftragte haftet im Allgemeinen für die gleiche Sorgfalt wie der Arbeitnehmer im Arbeitsverhältnis.
    \item Er haftet dem Auftraggeber für getreue und sorgfältige Ausführung des ihm übertragenen Geschäftes.
\end{itemize}

\subsubsection{Auftrag - Truepflicht}
\begin{itemize}
    \item Im Rahmen der Treuepflicht muss ein Beauftragter die Interessen seines Auftraggebers wahren. Der Beauftragte ist demnach bspw. verpflichtet, den Auftraggeber zu:
    \begin{itemize}
        \item beraten (auch Anweisungen des Auftraggebers kritisch zu hinterfragen)
        \item informieren (auch über Interessenskollisionen)
        \item Informationen geheim zu halten
    \end{itemize}
    \item Auch sollte der Auftragnehmer die Fähigkeiten und persönlichen Eigenschaften besitzen, um den Auftrag nach bestem Wissen und Gewissen erfüllen zu können.
\end{itemize}

\subsubsection{Auftrag - Widerruf/ Kündigung}
\begin{itemize}
    \item Der Auftrag kann von jedem Teile jederzeit widerrufen oder gekündigt werden.
    \item Erfolgt dies jedoch zur Unzeit, so ist der zurücktretende Teil zum Ersatze des dem anderen verursachten Schadens verpflichte
    \item Zwingendes Kündigungsrecht
    \item Eine Vertragsauflösung zur Unzeit liegt gemäss bundesgerichtlicher Rechtsprechung vor, wenn der Zeitpunkt der Kündigung besonders ungünstig ist und für den Vertragspartner besondere Nachteile mit sich bringt.
\end{itemize}