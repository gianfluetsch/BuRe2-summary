\section{IT Verträge}

\subsection{Eigenschaften}
\begin{itemize}
    \item Innominatvertrag (= gesetzlich nicht geregelt)
    \item enthalten oftmals Elemente des Auftrags und/oder des Werkvertrages, aber auch von anderen Vertragstypen
\end{itemize}

\subsection{Beispiele}
\begin{itemize}
    \item Entwicklung einer App, Erstellung einer Website oder Entwicklung einer Individualsoftware, Installationsund Reparaturarbeiten = Erfolg ist i.d.R. geschuldet, daher überwiegend \textbf{Werkvertrag}
    \item Nutzungsrechte bei Cloudverträgen = Können Elemente der \textbf{Miete} enthalten
    \item Selbstständige Planung, Beratung und Projektmanagementleistungen = Das sorgfältige Tätigwerden ist im Vordergrund, daher überwiegend \textbf{Auftrag}
    \item Erwerb einer Standardsoftware: Die Regelungen betreffend den \textbf{Kaufvertrag} kommen analog zur Anwendung. Individuell konzipierte IT-Produkte haben aber \textbf{werkvertraglichen Charakter}.
\end{itemize}

\subsection{Mängelrechte}
Bei Mängeln ist jeweils darauf zu achten, welche rechtlichen Grundlagen am ehesten zur Anwendung kommen, z.B.:

\begin{itemize}
    \item Bei Mängeln und Verzug bei \underline{kaufvertraglichen} Eigenschaften und die besonderen Bestimmungen zum Kaufvertrag (\textit{Wandelung, Minderung oder Ersatzlieferung})
    \item Bei Mängeln und Verzug bei \underline{werkvertraglichen} Eigenschaften und die besonderen Bestimmungen zum Werkvertrag (\textit{Wandelung, Minderung oder Nachbesserung})
    \item Bei Mängeln und Verzug bei \underline{auftragsrechtlichen} Eigenschaften und die besonderen Bestimmungen zum Auftrag (\textit{Schadenersatz})
\end{itemize}

\subsection{Erscheinungsformen}
Software-Lizenzvertrag, Wartungsvertrag, Dienstleistungsvertrag, Auftragsverhältnis, Werkvertrag, Vertrag für die Lieferung von integrierten Informatiksystemen, Systemintegrationsvertrag, Outsourcingvertrag, Vertrag für Erbringung von Hosting-Dienstleistungen, Vertrag für Konzeption und Realisierung einer WebApplikation, Vertrag Cloud Services, Hardwareverkaufsvertrag, etc.

\subsection{Regelungspunkte in den Verträgen}
Folgende Punkte sollten i.d.R. in IT-Verträgen geregelt werden:\\

\begin{minipage}{0.5\linewidth}
    \begin{enumerate}
        \item Genaue Umschreibung der vertraglichen Leistung
        \item Preis
        \item Nutzungsrechte
        \item Lizenzverträge
    \end{enumerate}
\end{minipage}
\begin{minipage}{0.5\linewidth}
    \begin{enumerate}
        \item Regelungen den Datenschutz betreffend
        \item Gewährleistungsrecht und Haftung
        \item Vertragsauflösung
    \end{enumerate}
\end{minipage}

\newpage

\subsection{Outsourcing-Vertrag}
Kann viele verschiedene Leistungen regeln. Es handelt sich daher bei diesem Vertrag um einen Innominatvertrag. Er stellt weiter oftmals einen gemischten Vertrag dar, der Elemente verschiedener Vertragstypen in sich vereinigt (Bspw. Miete, Pacht, Werkvertrag oder Auftrag).\\

Wichtige Phasen des Outsourcings sind die eigentliche \textbf{Auslagerung} der IT-Infrastruktur, der anschliessende \textbf{Betrieb} und die \textbf{Rückübertragung}. Über diese Phasen sollte der Vertrag
Auskunft geben können.

\subsection{Aufgaben}

\subsubsection{Software-Lizenzvertrag}
Ein Lizenzvertrag legt die Bedingungen zwischen dem Inhaber von Rechten an Geistigem Eigentum und dem Empfänger der Lizenz fest. Er beschreibt, wie und in welchem Umfang das Geistige Eigentum genutzt werden darf und regelt Fragen der Haftung, der Vertraulichkeit und des
Kündigungsrechts.
$\rightarrow$ \textbf{Mietvertrag} (könnte auch \underline{Kaufvertrag} sein, wenn unbegrenzte Lizenz gekauft wird)

\subsubsection{Wartungsvertrag}
Vertragsgegenstand bei einem Software-Wartungsvertrag ist der Erhalt und/oder die Wiederherstellung der Betriebsbereitschaft der Software, die Aktualisierung, Beratung sowie Pflege der Software.

$\rightarrow$ \textbf{Werkvertrag/ Auftrag} $\rightarrow$ je mehr eine Beratung einnimmt, desto eher Auftrag/ je weniger die Beratung einnimmt, desto eher Werkvertrag

\subsubsection{Softwareentwicklungsvertrag}
Der Software-Entwicklungsvertrag ist ein Spezialfall eines IT-Dienstleistungsvertrages. Darin verpflichtet sich der Entwickler zur Erstellung einer Software Applikation nach den konkreten Vorgaben des Bestellers.
$\rightarrow$ Grundsätzlich \textbf{Werkvertrag}, aber wenn es darum geht mal zu Versuchen ohne Garantie auf Funktionstauglichkeit $\rightarrow$ \textbf{Auftrag}

\subsubsection{Dienstleistungsvertrag für Planungen und Beratungen}
$\rightarrow$ Auftrag

\subsubsection{Verkauf von Domainnamen}
$\rightarrow$ Kaufvertrag

\subsubsection{Erstellen einer Website}
$\rightarrow$ Auftrag