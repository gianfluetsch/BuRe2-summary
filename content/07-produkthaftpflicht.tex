\section{Produkthaftpflicht}

\subsection{Ausservertragliche Haftung}

\subsubsection{Verschuldenshaftung}
\begin{itemize}
    \item Wer einem andern widerrechtlich Schaden zufügt, sei es mit Absicht, sei es aus Fahrlässigkeit, wird ihm zum Ersatze verpflichtet.
    \item Ebenso ist zum Ersatze verpflichtet, wer einem andern in einer gegen die guten Sitten verstossenden Weise absichtlich Schaden zufügt.
\end{itemize}

\paragraph{Voraussetzungen}
\begin{minipage}{0.5\linewidth}
    \begin{itemize}
        \item Schaden (unfreiwillige Vermögenseinbusse)
        \item Adäquater Kausalzusammenhang (Zusammenhang von schuldhaftem Verhalten und Schaden)
    \end{itemize}
\end{minipage}
\begin{minipage}{0.5\linewidth}
    \begin{itemize}
        \item Widerrechtlichkeit
        \item Verschulden (Fahrlässigkeit oder Vorsatz)
    \end{itemize}
\end{minipage}

\subsubsection{Kausalhaftung}
\textbf{Verschulden} ist \underline{nicht erforderlich} oder es wird vermutet.

\paragraph{Milde Kausalhaftung}
\begin{itemize}
    \item Verantwortliche muss für Schaden einstehen, wenn dieser durch eine unsorgfältige Handlung begründet worden ist.
    \item Haftung des Geschäftsherrn, Haftung des Tierhalters, Haftung des Familienoberhauptes und \textbf{Produktehaftpflicht}
\end{itemize}

\paragraph{Scharfe Kausalhaftung}
\begin{itemize}
    \item Haftpflicht besteht auch bei fehlender Ordnungswidrigkeit
    \item Grundeigentümer haftet für Verunreinigungen von Gewässern auch wenn ihm an einem Unfall keine Schuld trifft
    \item Werkeigentümerhaftung
\end{itemize}

\subsubsection{Gefährdungshaftung}
\begin{itemize}
    \item Qualifizierte Gefährdung durch Vorrichtung oder Tätigkeit
    \item Im Unterschied zur Haftung aus unerlaubter Handlung kommt es nicht auf die Widerrechtlichkeit der Handlung oder ein Verschulden des Schädigers an, sondern wird bspw. alleine schon durch den Betrieb (z.B. Motorfahrzeug) begründet.
    \item Eine potenzielle Gefährdung anderer ist \underline{unvermeidbar}.
    \item Bspw. Betrieb eines Motorfahrzeugs, Jagdbetrieb, Flugbetrieb, Betrieb einer Atomanlage (in Spezialgesetzen geregelt)
\end{itemize}

\newpage

\subsubsection{Vertragliche Haftung}
\begin{itemize}
    \item Der Verkäufer haftet dem Käufer sowohl für die zugesicherten Eigenschaften als auch dafür, dass die Sache nicht körperliche oder rechtliche Mängel habe, die ihren Wert oder ihre Tauglichkeit zu dem vorausgesetzten Gebrauche aufheben oder erheblich mindern.
    \item Er haftet auch dann, wenn er die Mängel nicht gekannt hat.
\end{itemize}

\paragraph{Voraussetzungen}
\begin{minipage}{0.5\linewidth}
    \begin{itemize}
        \item Schaden
        \item Vertragsverletzung
    \end{itemize}
\end{minipage}
\begin{minipage}{0.5\linewidth}
    \begin{itemize}
        \item Verschulden
        \item Kausalzusammenhang
    \end{itemize}
\end{minipage}

\subsection{Produktehaftpflicht}

\subsubsection{Grundsatz}
\begin{itemize}
    \item Die herstellende Person (Herstellerin) haftet für den Schaden, wenn ein \underline{fehlerhaftes} Produkt (Kausalhaftung) dazu führt, dass:
    \item eine Person getötet oder verletzt wird; (Personenschaden)
    \item eine Sache beschädigt oder zerstört wird, die nach ihrer Art gewöhnlich zum privaten Gebrauch oder Verbrauch bestimmt und vom Geschädigten hauptsächlich privat verwendet worden ist. (Sachschaden)
    \item Die Herstellerin \underline{haftet \textbf{nicht}} für den Schaden am fehlerhaften Produkt.
    \item Führt bspw. ein fehlerhaftes Kletterseil zu einem Sportunfall, geht es vorliegend nicht um den Ersatz des fehlerhaften Produktes.
    \begin{itemize}
        \item Bei Verkäufer kann nur Ersatz des Seils gefordert werden
        \item Weitere Schaden (Verletzung etc.) muss beim Hersteller gefordert werden
    \end{itemize}
\end{itemize}

\subsubsection{Begriff des Fehlers}
Ein Produkt ist \underline{fehlerhaft}, wenn es nicht die Sicherheit bietet, die man \underline{unter Berücksichtigung aller Umstände} \underline{zu erwarten berechtigt ist}, insbesondere sind zu berücksichtigen:
\begin{itemize}
    \item die \underline{Art und Weise}, in der es dem Publikum präsentiert wird
    \item der Gebrauch, mit dem \underline{vernünftigerweise} gerechnet werden kann
    \item der \underline{Zeitpunkt}, in dem es in Verkehr gebracht wurde.
    \item Ein Produkt ist nicht allein deshalb fehlerhaft, weil später ein verbessertes Produkt in Verkehr gebracht wurde.
\end{itemize}

\newpage

\subsubsection{Voraussetzungen}
\label{Produkthaftpflicht:checklist}
Folgende Voraussetzungen müssen erfüllt sein, damit der Hersteller haftet:\\

\fbox{\parbox{\linewidth}{
    \begin{enumerate}
        \item Er ist ein \textbf{Hersteller / Importeur / Händler}
        \item Es handelt sich um ein \textbf{Produkt}
        \item Produkt ist \textbf{fehlerhaft}
        \item Es verursachte einen Schaden und der \textbf{Schaden} ist grösser als der Selbstbehalt von CHF $900.00$
        \begin{itemize}
            \item Selbstbehalt ist fix bei $900$ CHF (Schaden von $1000 - 900 = 100$ Schadenersatz)
        \end{itemize}
        \item Zwischen fehlerhaften Produkt und Schaden besteht ein \textbf{adäquater Kausalzusammenhang}
        \begin{itemize}
            \item \textit{adäquater Kausalzusammenhang = fehlerhaftes Produkt hat den Schaden zu verantworten}
        \end{itemize}
    \end{enumerate}
}}

\vspace{5mm}

\textbf{Merke}: Vereinbarungen, welche die Haftpflicht nach diesem Gesetz gegenüber dem Geschädigten beschränken oder wegbedingen, sind nichtig.
\textit{\textcolor{OSTPink}{Verjährungsfrist}} beträgt \textbf{\underline{drei Jahre}} und die \textit{\textcolor{OSTPink}{Verwirkungsfrist}} \textbf{\underline{zehn Jahre}}.

\begin{itemize}
    \item \textbf{\textcolor{OSTPink}{Verjährung}}
    \begin{itemize}
        \item Anspruch kann vor Gericht eingefordert werden, Gegenseite kann aber Verjährung geltend machen (3 Jahre und 1 Tag)
    \end{itemize}
    \item \textbf{\textcolor{OSTPink}{Verwirkungsfrist}}
    \begin{itemize}
        \item Richter sagt, dass Ansprüche nicht mehr geltend gemacht werden (z.B. 10 Jahre und 1 Tag)
    \end{itemize}
\end{itemize}

\subsubsection{Anwendungsfälle}
\begin{itemize}
    \item Fehlerhaftigkeit von Geräten oder Maschinen oder andere Arbeitsmittel auf Baustellen oder in der Industrie
    \item Fehlerhafte Glasskanne explodiert bei Überhitzung
    \item Kletterseil hält nicht bei vorgesehener Belastungsgrenze
    \item Rahmenbruch bei einem Fahrrad
    \item Explosion einer Sektflasche
    \item Notebook geht, aufgrund eines fehlerhaften Akkus, in Flammen auf.
    \item Greifer einer Anlage griffen aufgrund einer fehlerhaften Dimensionierung nicht.
    \item Brüchige Schrauben, falscher Stahl, gefährliche Fehlkonstruktionen, missverständliche Bedienungsanleitungen, fehlende Sicherheitshinweise, mit Bakterien verseuchte oder verunreinigte Lebensmittel etc.
\end{itemize}

\newpage

\subsection{Aufgaben}

\begin{itemize}
    \item Gibt es bei Einzelarbeitsverträgen auch Formvorschriften zu beachten?
    \begin{itemize}
        \item \textit{Es gibt im OR noch besondere Einzelarbeitsverträge, z.B. dem Lehrvertrag}
    \end{itemize}
    \item Die Vertragspunkte eines Gesamtarbeitsvertrages dürfen einen Arbeitnehmer keinesfalls schlechter stellen als die Vorschriften im Obligationenrecht, korrekt?
    \begin{itemize}
        \item \textit{Es muss geschaut werden ob die Bestimmungen im Obligationenrecht zwingend sind -> dann darf der Arbeitnehmer nicht schlechter gestellt werden, sonst darf Arbeitnehmer schlechter gestellt werden.}
    \end{itemize}
    \item Welchen Anforderungen muss ein Arbeitszeugnis grundsätzlich entsprechen? Wo sehen Sie allenfalls Zielkonflikte?
    \begin{itemize}
        \item \textit{Zielkonflikt bei Wohlwollend und Wahrheitsgetreu oder Klarheit und Vollständigkeit}
    \end{itemize}
    \item Bei der Haftung des Arbeitnehmers gemäss Art. 321e OR kann man den Arbeitnehmer auch mehr in die Pflicht nehmen, als es das OR vorsieht, korrekt?
    \begin{itemize}
        \item \textit{Nein, weil es sich um eine relativ besimmende Massnahme des OR handelts}
    \end{itemize}
    \item Die Überstunden kann man insgesamt wegbedingen, die Überzeit hingegen nicht, korrekt?
    \begin{itemize}
        \item \textit{Überzeit kann ausser bei gewissen Ausnahmen nicht wegbedingt werden, Überstunden können wegbedingt werden}
    \end{itemize}
    \item Wenn der Arbeitnehmer in einer Sperrfrist kündigt, ist die Kündigung nichtig, korrekt?
    \begin{itemize}
        \item \textit{Arbeitnehmer kann immer kündigen -> somit ist die Kündigung korrekt/ Arbeitgeber darf in Sperrfrist nicht kündigen (ausser man ist noch in der Probezeit)}
    \end{itemize}
    \item Bei der missbräuchlichen Kündigung muss man einen gewissen Ablauf berücksichtigen, welchen?
    \begin{itemize}
        \item \textit{In Kündigungsfrist -> schriftliche Einsprache, spätestens 180 Tage nach Ende Arbeitszeit muss Klage eingereicht werden}
    \end{itemize}
    \item Ein Werkvertrag kann formfrei geschlossen werden, korrekt? $\rightarrow$ \textit{Korrekt}
    \item Wie muss der Besteller vorgehen, wenn er das Werk entgegennimmt? Was muss er tun, wenn er Mängel entdeckt? Welche Möglichkeiten hat er dann?
    \begin{itemize}
        \item Werk direkt bei Erhalt prüfen und falls es (offene) Mängel gibt diese festhalten und hat Anrecht auf Wandlung oder Nachbesserung (Verhältnissmässig was das beste ist)
    \end{itemize}
    \item Handelt es sich bei der Reparatur eines Klaviers um einen Auftrag oder um einen Werkvertrag? Begründen Sie kurz anhand der Ihnen bekannten Abgrenzungskriterien.
    \begin{itemize}
        \item Um einen Werkvertrag, da kein Erfolg geschuldet ist (man hat keine Garantie, dass das Klavier repariert werden kann). I.d.R sind solche Reparaturen ein Werkvertrag (auch Reifenwechsel, etc.)
    \end{itemize}
    \item Auch nach mehrfacher Ermahnung verzögert der Unternehmer die Bauarbeiten unbegründet. Was kann der Besteller hier machen?
    \begin{itemize}
        \item Der Besteller kann von der Bestellung zurücktreten oder man könnte dem Unternehmen androhen auf Kosten des Unternehmers die Bauarbeiten mit einem anderen Unternehmen weiterzuführen
    \end{itemize}
    \item Der Anwalt weigert sich kurz vor der Verhandlung, das Mandat niederzulegen. Er ist der Meinung, dass er bereits alle Vorbereitungshandlungen gemacht habe und so kurzfristig keinen neuen Klienten mehr finden würde. Der Klient ist da anderer Meinung. Wer hat Recht?
    \begin{itemize}
        \item Auftrag, da als Anwalt kein Erfolg garantiert werden kann.
        \item Kündigungsrecht/ Widerrufsrecht $\rightarrow$ Klient kann Vertrag mit Anwalt kündigen, Anwalt könnte aber Anrecht auf Schadenersatz verlangen
    \end{itemize}
\end{itemize}

Roman kauft bei einem Sportfachgeschäft ein Kletterseil. Dieses soll gemäss Angaben des Herstellers (Alternativ des Verkäufers) ca. 1000 Kilogramm Gewicht tragen können. Leider reisst das Seil aber beim Klettern bereits nach einer Last von 50 Kilogramm ab. Bei der Begutachtung und der Ermittlung des Sachverhalts stellt sich heraus, dass dieses Seil:

\begin{itemize}
    \item einen Materialfehler hat;
    \begin{itemize}
        \item (Checkliste \ref{Produkthaftpflicht:checklist} durchgehen) $\rightarrow$ klarer Fall von Produkthaftpflicht
    \end{itemize}
    \item bereits beim Kauf beschädigt worden sein muss;
    \begin{itemize}
        \item Produkthaftpflicht kein Thema (kein Fehler vom Hersteller) $\rightarrow$ Fehler bei Verkäufer
    \end{itemize}
    \item Irrtümlicherweise vom Verkäufer empfohlen worden ist und gar kein Kletterseil war.
    \begin{itemize}
        \item Fehlerhaftigkeit nicht gegeben (Hersteller hat kein Kletterseil hergestellt, Verkäufer hat es aber als Kletterseil verkauft)
        \item Geschädigter kann gegen Verkäufer (Geschäft) vorgehen oder sogar direkt gegen den Verkäufer vorgehen
    \end{itemize}
\end{itemize}