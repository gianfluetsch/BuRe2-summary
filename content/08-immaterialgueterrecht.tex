\section{Immaterialgüterrecht}

\subsection{Immaterialgüterrecht}
Immaterialgüterrechte sind Schutzrechte \underline{technischer} (technische Erfindungen), ästhetischer (Literatur, Design, Musik) oder kennzeichnungsrechtlicher Natur (Marke, Unternehmenskennzeichen), die ein Exklusivrecht vermitteln.
Schutzrechte sind insbesondere:\\

\begin{minipage}{0.5\linewidth}
    \begin{itemize}
        \item Markenschutz
        \item Patentschutz
    \end{itemize}
\end{minipage}
\begin{minipage}{0.5\linewidth}
    \begin{itemize}
        \item Designschutz
        \item Urheberrecht
    \end{itemize}
\end{minipage}

\subsection{Markenschutz/ Markenrecht}
Eine Marke ist ein \underline{geschütztes Kennzeichen}, mit dem der Inhaber derselben seine Waren und Dienstleistungen von denen anderer Unternehmen unterscheiden kann. Grundsätzlich können \underline{alle grafisch darstellbaren Zeichen} Marken sein: z.B. Wörter, Buchstabenkombinationen, Zahlenkombinationen, bildliche Darstellungen, dreidimensionale Formen, Slogans, Kombinationen dieser Elemente, oder auch aus Tonfolgen bestehen.

\subsubsection{Beginn des Schutzes}
Eintragung der Marke ins Schweizerische Markenregister (Swissreg). Das Markenrecht steht demjenigen zu, der die Marke zuerst hinterlegt.

\subsubsection{Markentypen}
Wortmarken, Bildmarken, Kombinationen, dreidimensionale Marken (Mercedes - Stern), akustische Marken (Jingles), Positionsmarken (Vans), Farbmarken (Gelb der Post), Bewegungsmarken

\subsubsection{Schutzvoraussetzungen}
Ausreichende \underline{Unterscheidbarkeit}: Massgebend für die Beurteilung der Unterscheidungskraft ist der \underline{Gesamteindruck}, den eine Marke hinterlässt.

Als Marke \underline{nicht schutzfähig} sind:
\begin{itemize}
    \item \textbf{Absolute Ausschlussgründe}
    \begin{itemize}
        \item Einfache Zeichen, Abkürzungen, Sachangaben und Wappen. Zudem darf die Marke nicht beschreibend (Angaben zu Beschaffenheit, Qualität, Art oder Ort der Herstellung, Bestimmung oder Preis der Ware) oder täuschend sein (Herkunft, Qualität oder Beschaffenheit, z.B. Beltina Suisse für Fahrräder aus Frankreich) und nicht gegen die öffentliche Ordnung verstossen (Sexuell anstössig oder rassistisch).
    \end{itemize}
    \item \textbf{Relative Ausschlussgründe}
    \begin{itemize}
        \item Vom Markenschutz ausgeschlossen sind ausserdem Zeichen, die
        \begin{itemize}
            \item mit einer älteren Marke identisch \& für die gleichen Waren oder Dienstleistungen bestimmt sind
            \item mit einer älteren Marke identisch \& für gleichartige Waren oder Dienstleistungen bestimmt sind, so dass sich daraus eine Verwechslungsgefahr ergibt
            \item einer älteren Marke ähnlich \& für gleiche oder gleichartige Waren oder Dienstleistungen bestimmt sind, so dass sich daraus eine Verwechslungsgefahr ergibt.
        \end{itemize}
    \end{itemize}
\end{itemize}

\subsubsection{Dauer}
Eine Marke ist vom Anmeldedatum an für jeweils \underline{zehn Jahre} geschützt. Der Schutz kann beliebig oft um weitere zehn Jahre verlängert werden.

\subsubsection{Rechtsschutz/ Markenrechtsverletzung}
\paragraph{Klagen nach Zivilrecht}
\begin{itemize}
    \item Feststellungsklage
    \item Leistungsklagen: Unterlassung oder Beseitigung
    \item Einziehung im Zivilverfahren: Das Gericht kann die Einziehung und Verwertung oder Vernichtung der widerrechtlich hergestellten Gegenstände anordnen.
    \item Vorsorgliche Massnahmen
    \item Klage auf Schadenersatz, Genugtuung oder Gewinnherausgabe nach OR
\end{itemize}

\paragraph{Klagen nach Strafrecht}

\begin{itemize}
    \item Markenrechtsverletzung
    \item Betrügerischer Markengebrauch
    \item Gebrauch unzutreffender Herkunftsangaben
\end{itemize}

\subsection{Patentschutz}
Ein Patent ist ein Schutzrecht für eine \underline{neue gewerblich anwendbare Erfindungen}. Für eine solche Erfindung wird ein Erfindungspatent erteilt.
Eine Erfindung gilt als neu, wenn sie \underline{nicht zum Stand der Technik} gehört. Den Stand der Technik bildet alles, was vor dem Anmelde- oder dem Prioritätsdatum der Öffentlichkeit durch schriftliche oder mündliche Beschreibung, durch Benützung oder in sonstiger Weise zugänglich gemacht worden ist. \textit{Unbedingt Erfindung bis zur Anmeldung geheim behalten!!!}

Die \textbf{Erfindung} kann ein \textit{Produkt} oder ein \textit{Verfahren} betreffen:
\begin{itemize}
    \item Als \textbf{Produkte} gelten z.B. Erzeugnisse wie Waren und Werkzeuge, Vorrichtungen wie Produktionsanlagen und Maschinen, Materialien wie chemische Substanzen oder textile Stoffe.
    \item Als \textbf{Verfahren} gelten zielgerichtete Handlungen wie Herstellungsverfahren (Arbeits- oder Produktionsschritte, um ein Produkt zu fertigen), Steuerungsverfahren (Prozessschritte bei der Verwendung einer Vorrichtung oder Maschine) oder Messverfahren.
\end{itemize}

\subsubsection{Beginn des Schutzes}
Wer ein Erfindungspatent erlangen will, hat beim Eidgenössischen Institut für Geistiges Eigentum (IGE) ein Patentgesuch einzureichen.
Das Patent wird vom IGE durch Eintragung ins Patentregister erteilt.

\subsubsection{Schutzvoraussetzungen}
\begin{itemize}
    \item Die Erfindung ist \underline{\textit{neu}}
    \item Die Erfindung ist \underline{\textit{erfinderisch}}
    \item Die Erfindung ist \underline{\textit{gewerblich nutzbar, anwendbar und wiederholbar}}
\end{itemize}

\subsubsection{Dauer}
Das Patent kann längstens bis zum Ablauf von \underline{20 Jahren} seit dem Datum der Anmeldung dauern.

\subsubsection{Nicht Patentierbar}
\begin{minipage}{0.5\linewidth}
    \begin{itemize}
        \item Abstrakte Ideen ohne konkrete technische Lösungsschritte
        \item Spielregeln und Lehrmethoden
        \item diagnostische, therapeutische und chirurgische Verfahren
    \end{itemize}
\end{minipage}
\begin{minipage}{0.5\linewidth}
    \begin{itemize}
        \item Pflanzensorten, Tierrassen und biologische Verfahren zur Züchtung von Pflanzen oder Tieren
        \item Computerprogramme als solche (sind durch \textit{Urheberrecht} geschützt)
        \item Erfindungen, deren Verwertung gegen die öffentliche Ordnung oder die guten Sitten verstösst
    \end{itemize}
\end{minipage}

\subsubsection{Rechtsschutz/ Patentrechtsverletzung}

\paragraph{Klagen nach Zivilrecht}
\begin{itemize}
    \item Feststellungsklage
    \item Leistungsklagen: Unterlassung oder Beseitigung
    \item Einziehung im Zivilverfahren: Das Gericht kann die Einziehung und Verwertung oder Vernichtung der widerrechtlich hergestellten Gegenstände anordnen.
    \item Vorsorgliche Massnahmen
    \item Klage auf Schadenersatz, Genugtuung oder Gewinnherausgabe nach OR
\end{itemize}

\paragraph{Klagen nach Strafrecht}
\begin{itemize}
    \item Patentverletzung
\end{itemize}

\subsection{Designschutz/ Designrecht}
Ein Design ist die \underline{kreative Gestaltung eines Gegenstands}. Durch die Anordnung von Linien, Konturen, Farben, Flächen oder durch das verwendete Material erhält es \underline{seinen eigenen Charakter}. Geschützt werden können \underline{zweidimensionale Designs}, wie z.B. die Gestaltung eines Stoffmusters, eines Uhrenzifferblatts oder einer Flaschenetikette, sowie \underline{dreidimensionale Designs}, also z.B. die Form einer Zahnbürste, einer Lampe oder eines Stuhls. Das Designgesetz soll solche Erzeugnisse (Form, die äussere Gestaltung eines Gegenstandes) schützen.

\subsubsection{Beginn des Schutzes}
Das Designrecht entsteht mit der \underline{Eintragung im Design Register} (Register). Ein Design gilt als hinterlegt, wenn beim Eidgenössischen Institut für Geistiges Eigentum (IGE) ein \underline{Eintragungsgesuch} eingereicht worden ist.

\subsubsection{Schutzvoraussetzungen}
Ein Design muss folgende Bedingungen erfüllen, damit es geschützt werden kann:
\begin{itemize}
    \item Es muss \textbf{\underline{neu}} sein. Neu ist ein Design, wenn vor der Hinterlegung kein anderes identisches oder ähnliches Design veröffentlicht worden ist.
    \item Es muss \textbf{\underline{Eigenart}} aufweisen, das heisst sich genügend von bestehenden Designs unterscheiden
    \item Es darf \textbf{\underline{weder gesetzeswidrig noch anstössiger Natur}} sein.
\end{itemize}

\subsubsection{Dauer}
Insgesamt \underline{25 Jahre} (5 Jahre + viermaliger Verlängerung von je 5 Jahren)

\subsubsection{Design vs. Patent}
\begin{itemize}
    \item \textit{Design}: ästhetische Wirkung
    \item \textit{Patent}: technische Funktion
\end{itemize}

\subsubsection{Designrechtsverletzungen}
\paragraph{Klagen nach Zivilrecht}
\begin{itemize}
    \item Feststellungsklage
    \item Abtretungsklage: Wer ein besseres Recht geltend macht, kann gegen die Rechtsinhaberin auf Abtretung des Designrechts klagen.
    \item Leistungsklagen: Verletzung verbieten oder beseitigen
    \item Einziehung im Zivilverfahren: Das Gericht kann die Einziehung und Verwertung oder Vernichtung der widerrechtlich hergestellten Gegenstände anordnen.
    \item Vorsorgliche Massnahmen: Beweissicherung, Ermittlung der Herkunft
    \item Klage auf Schadenersatz, Genugtuung oder Gewinnherausgabe nach OR
\end{itemize}

\paragraph{Klagen nach Strafrecht}
\begin{itemize}
    \item Designrechtsverletzung
\end{itemize}

\subsection{Urheberrecht}
Das Urheberrecht verfolgt den Zweck, Werkschaffende (Künstler, Autoren, Komponisten, Maler oder Regisseure) zu schützen. Es soll in den Händen der Werkschaffenden liegen, ob, wann und wie ihre Werke verwendet werden dürfen. Die Bestimmungen des Urheberrechts klären zudem, wann dieser Schutz entsteht und wie ein geschütztes Werk genutzt werden kann.

\subsubsection{Beginn des Schutzes}
Der Urheberrechtsschutz entsteht \underline{automatisch}. Es gibt kein Register. Das Werk ist geschützt, sobald es geschaffen wurde. Hinweise wie z. B. © oder «Copyright» am Werk müssen nicht angebracht werden, damit es geschützt ist. Es gilt die Alterspriorität. Diese richtet sich nach dem erstmaligen Bekanntwerden.

\subsubsection{Schutzvoraussetzung}
Geistige Schöpfung mit individuellem Charakter

\subsubsection{Dauer}
Urheberrechtliche Werke 70 Jahre (bei Computerprogrammen: 50 Jahre) über den Tod des Urhebers hinaus. \\
\textit{Ausnahme}: Schutz von Fotografien ohne individuellen Charakter 50 Jahre ab Herstellung.

\subsubsection{Urheberrechtsverletzungen}
Das URG sieht bei Urheberrechtsverletzungen \underline{zwei Arten von rechtlichen Schritten} vor. Es sind dies:

\paragraph{Klagen nach Zivilrecht}
\begin{itemize}
    \item Feststellungsklage
    \item Leistungsklagen: Verletzung verbieten oder beseitigen; Bekanntgaberechte
    \item Einziehung im Zivilverfahren: Das Gericht kann die Einziehung und Verwertung oder Vernichtung der widerrechtlich hergestellten Gegenstände anordnen.
    \item Vorsorgliche Massnahmen: Beweissicherung, Ermittlung der Herkunft
    \item Klage auf Schadenersatz, Genugtuung oder Gewinnherausgabe nach OR
\end{itemize}

\paragraph{Klagen nach Strafrecht}
Anders als bei den zivilrechtlichen Klagen, wo eine rechtswidrig handelnde Person auch bei fahrlässigem Verhalten verurteilt werden kann, macht sich nach den strafrechtlichen Vorschriften nur strafbar, wer einen Verstoss mit \underline{Vorsatz} begeht.

\begin{itemize}
    \item Urheberrechtsverletzung
    \item Unterlassung der Quellenangabe
\end{itemize}

\subsection{Voraussetzungen Urheberrechtsverletzung}
\begin{itemize}
    \item Vorsatz
    \begin{itemize}
        \item Von einem \underline{direkten Vorsatz} spricht man, wenn sich der Täter sicher ist, dass der Deliktserfolg eintritt und er diesen auch anstrebt (willentlich und wissentlich handeln). \underline{Eventualvorsätzlich} handelt ein Täter, wenn er den Deliktserfolg für möglich hält und ihn in Kauf nimmt.
    \end{itemize}
    \item Widerrechtlichkeit
    \begin{itemize}
        \item Die Verwendung eines geschützten Werks ist dann widerrechtlich, wenn
        \begin{itemize}
            \item der Nutzer \underline{nicht der Urheber} des Werks ist,
            \item ihm \underline{keinerlei Rechte am Werk} übertragen worden sind,
            \item die Verwendung des Werks \underline{nicht durch eine Ausnahme}, bspw. im Sinne des Eigengebrauchs, \underline{gerechtfertigt} werden kann und
            \item \underline{keine Einwilligung des Rechteinhabers} vorliegt.
        \end{itemize}
        \item z.B. Wenn Fotograf Fotos macht und diese dem Fotografen abgekauft werden, darf man diese aber nicht  uneingeschränkt nutzen -> Fotograf ist weiterhin der Urheber der Fotos und dieser muss gefragt werden ob Fotos z.B. Kommerziell genutzt werden dürfen (z.B. Fotos auf Insta !!!)
    \end{itemize}
\end{itemize}